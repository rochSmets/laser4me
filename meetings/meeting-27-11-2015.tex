\documentclass{report}

\usepackage{graphicx}
\usepackage{amsmath}
\usepackage{amsfonts}
\usepackage{amssymb}
\usepackage[utf8]{inputenc}

\setlength{\textwidth}{16.0cm}
\setlength{\textheight}{22.0cm}
\setlength{\oddsidemargin}{0.0cm}
\setlength{\evensidemargin}{0.0cm}
\setlength{\topmargin}{-1.0cm}

\renewcommand{\baselinestretch}{1.2}

\begin{document}

{\it
\noindent
r\'eunion du 27 Novembre 2015}

\bigskip

\begin{center}
{\large
\bf
Pr\'eparation de LMJ/PETAL+ (2017)\\[0.2cm]
}
{\it }
\end{center}

\bigskip

\noindent
{\bf Les LASERS LMJ/PETAL+}

\noindent $\bullet$ Pour avoir les 4 Quads en tache D, on shift en septembre 2017 (en avril, 2 taches D \& 2 taches E).\\
\noindent $\bullet$ La synchronisation entre QUAD n'est pas mécanique comme pour LULI2000.\\
\noindent $\bullet$ LMJ : 12 kJ / 5 ns --- Petal+ : 0.7 kJ / 0.7 ps\\
\noindent $\rightarrow$ pas de problemes pour LMJ.\\
\noindent $\rightarrow$ pour Petal, 0.7 kJ est vraiment le max... à voir si l'on peut se satisfaire de moins... $\sim$100J.\\


\noindent
{\bf Spectromètre DP1}

\noindent $\bullet$ Pour DP1, les 2 taches doivent tenir sur un carré de 3x3mm\\
\noindent $\rightarrow$ est-ce cohérent avec les previsions FCI2 ?\\


\noindent
{\bf CRACC : Cassette radiochromique en centre chambre}

\noindent $\bullet$ Films RCF de diamètre 95mm. Les angles possibles sont -30$^o$, -15$^o$, 0$^o$, +15$^o$.\\
\noindent $\bullet$ Composition des stack RCF modulable.\\
\noindent $\rightarrow$ il faudra les préciser avant Janvier 2017.\\
\noindent $\bullet$ SEPAGE a aussi 2 paraboles thomson.\\
\noindent $\rightarrow$ en a-t-on besoin ? Emmanuel d'Humière a fait des simulations PIC des spectres de protons \& divergences de faisceau.\\
\noindent $\bullet$ Besoin de conna\^{\i}tre la composition de la cassette (quels films, quelles énergies) 4 mois avant les tirs.\\
\noindent $\bullet$ Transmettre les infos sur les RCF souhaités afin d'orienter les qualifs SEPAGE.\\


\noindent
{\bf Cibles}

\noindent $\bullet$ Besoin d'un état de surface des cibles suffisament bon pour etre réfléchissant.\\
\noindent $\bullet$ Penser à la redondances pour les cibles.\\
\noindent $\rightarrow$ avoir + de 3 cibles pour la premiere campagne. Il est possible d'avoir tout les tires en 2017.\\
\noindent $\bullet$ Donner les specifications des cibles en avril 2016.\\
\noindent $\bullet$ Livrer les cibles en avril 2017.\\


\noindent
{\bf Organisation}

\noindent $\bullet$ tir en alternance : au plus 1 tout les 2 jours.\\
\noindent $\rightarrow$ permet d'avoir le temps de "developper" les RCF (quelques heures)\\
\noindent $\rightarrow$ rectifier la config du tir d'apres.\\
\noindent $\bullet$ La config doit être faite sous CAO pour septembre 2016 : du boulot pour Julien !\\

\noindent $\Rightarrow$ \underline{Prochaine reunion de suivi le vendredi 3 juin apres-midi.}\\


\end{document}
