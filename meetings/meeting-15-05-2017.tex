\documentclass{report}

\usepackage{graphicx}
\usepackage{amsmath}
\usepackage{amsfonts}
\usepackage{amssymb}
\usepackage[utf8]{inputenc}

\setlength{\textwidth}{16.0cm}
\setlength{\textheight}{22.0cm}
\setlength{\oddsidemargin}{0.0cm}
\setlength{\evensidemargin}{0.0cm}
\setlength{\topmargin}{-1.0cm}

\renewcommand{\baselinestretch}{1.2}

\begin{document}

{\it
\noindent
r\'eunion du 15 Mai 2017}

\bigskip

\begin{center}
{\large
\bf
Pr\'eparation de LMJ/PETAL+ (2017)\\[0.2cm]
}
{\it }
\end{center}

\bigskip

\noindent
{\bf Les LASERS LMJ/PETAL+}

\noindent $\bullet$ On pourrait esp\'erer aller au del\`a de 0.7 kJ sur PETAL+.\\
\noindent $\rightarrow$ Car il va falloir remplacer les miroirs de toute fa\c{c}on.\\


\noindent
{\bf Spectromètre DP1}

\noindent $\bullet$ Decallage de la manip pour avoir DP1.\\
\noindent $\rightarrow$ 3 tirs en decembre 2017 sans DP1 pour callibrer.\\
\noindent $\rightarrow$ 3 tirs dernier trimestre 2018 avec DP1.\\


\noindent
{\bf Spectromètre DMX}

\noindent $\bullet$ Avec un Z plus petit, pourquoi pas ?\\
\noindent $\rightarrow$ 20 voix, mais aussi 16 avec mini-DMX.\\


\noindent
{\bf CRACC : Cassette radiochromique en centre chambre}

\bigskip


\noindent
{\bf Cibles}

\noindent $\bullet$ On pourrait viser des cibles plus fines, ou \`a Z plus petit.\\
\noindent $\rightarrow$ Il y a alors moins d'ablation, mais elle est mangée plus vite par le laser.\\
\noindent $\rightarrow$ On a alors moins de diffusion, donc une image plus nette.\\
\noindent $\bullet$ pour les temps court, 3 ou 4 $\mu$m d'AU.\\
\noindent $\bullet$ Ou aussi essayer le Ti (Z plus petit).\\
\noindent $\bullet$ Autre solution : cibles bi-matériaux.\\
\noindent $\bullet$ On vise de préparer 3 cibles de 50 $\mu$m, et 3 de 100 $\mu$m.\\


\noindent
{\bf Organisation}

\noindent $\bullet$ Les cibles LMJ devront \^etre mont\'ees et livr\'ees au LIE en Sep. 2017..\\
\noindent $\rightarrow$ Les cibles PETAL+ pourront arriver mi-Oct. 2017 (pour savoir ce que l'on monte).\\
\noindent $\bullet$ La manip. avec DP1 pourrait \^etre dernier trimestre 2018.\\


\noindent $\Rightarrow$ \underline{Prochaine reunion de suivi...}\\

\end{document}
