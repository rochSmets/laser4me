\documentclass{report}

\usepackage{graphicx}
\usepackage{amsmath}
\usepackage{amsfonts}
\usepackage{amssymb}
\usepackage[utf8]{inputenc}

\setlength{\textwidth}{16.0cm}
\setlength{\textheight}{22.0cm}
\setlength{\oddsidemargin}{0.0cm}
\setlength{\evensidemargin}{0.0cm}
\setlength{\topmargin}{-1.0cm}

\renewcommand{\baselinestretch}{1.2}

\begin{document}

{\it
\noindent
r\'eunion du 13 Mars 2017}

\bigskip

\begin{center}
{\large
\bf
Pr\'eparation de LMJ/PETAL+ (2017)\\[0.2cm]
}
{\it }
\end{center}

\bigskip

\noindent
{\bf Les LASERS LMJ/PETAL+}

\noindent $\bullet$ moins de 300 J car dommages sur les miroirs (15\%, allant \`a 25\%) \\


\noindent
{\bf Spectromètre DP1}

\noindent $\bullet$ Pour DP1, les 2 taches doivent tenir sur un carr\'e de 3$\times$3 mm\\
\noindent $\rightarrow$ 10$\times$10 mm pour DP3\\


\noindent
{\bf CRACC : Cassette radiochromique en centre chambre}


\noindent
{\bf Cibles}

\noindent $\bullet$ 5 $\mu$m de Au pour un TNSA \`a 15 MeV (extrapolations donn\'ees Omega)\\
\noindent $\rightarrow$ 3 cm entre cible \& TCC, puis 10 cm jusqu'au stack RCF\\
\noindent $\bullet$ 50 $\mu$m de CH + 5 $\mu$m de Al (3$\times$3 mm)\\
\noindent $\rightarrow$ non autoporteur... donc besoin d'un support aluminis\'e\\
\noindent $\bullet$ SMCI va d\'ecouper les 20 cibles ``GoodFellow'' : pas besoin de cadres\\
\noindent $\bullet$ LMJ sera en centre chambre, et PETAL+ sera deport\'e\\
\noindent $\bullet$ pour les 3 tirs, 6 cibles : 3 CH + 10 Au\\
\noindent $\rightarrow$ cibles livr\'ees au LIE, S36\\
\noindent $\rightarrow$ campagne S48\\
\noindent $\bullet$ Ga\"el doit redimensionner les cibles : 16$\times$16 mm \& non 8$\times$8 mm\\


\noindent
{\bf Organisation}

\noindent $\bullet$ Alain Grisollet : 01 6926 5247\\


\noindent $\Rightarrow$ \underline{Prochaine reunion de suivi le lundi 15 mai apres-midi.}\\


\end{document}
